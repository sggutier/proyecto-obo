El sistema final, a diferencia de un sistema de punto de venta común, contará con funcionalidades más específicas a un estudio de fotografía, como lo es el control durante los procesos aceptación de los pedidos y la entrega de los resultados.

\subsection{Funciones del producto}
\begin{itemize}
\item Control del inventario de materiales para fotografía y video.
\item Control del inventario de productos en venta.
\item Control de las ventas realizadas.
\item Visualización del estado de los pedidos (Dinero abonado, qué área lo lleva a cabo, quién lo lleva a cabo).
\item Agregar sesiones fotográficas o de video por contrato a un calendario junto con su información y quién irá a atenderlo.
\item Control del ingreso y salida de los empleados, así como su disponibilidad.
\end{itemize}

\subsection{Características de los usuarios}

Debido a las múltiples funciones que se deben realizar el sistema solo será atendido por dos tipos de usuarios:
\begin{itemize}
\item Personal: Con nivel escolar promedio de preparatoria, contar con conocimientos de uso de sistemas de edición de imagen y video, mínimo conocimiento básico en administración y atención a clientes.
\item Administrador: Igual que el personal pero con mayor autoridad y responsabilidades.	
\end{itemize}

\subsection{Restricciones}
\begin{itemize}
\item El sistema será ejecutado a través de un navegador web.
\item Debe funcionar en equipos con 2.0 GHz de procesador y 2 GB de RAM.
\item Debe ser intuitivo al grado de poder operarlo sin capacitación.
\item Debe ser capaz de permitir la exportación de la información a un archivo Excel
\item Contar con autentificación de usuario.
\item La información deberá ser almacenada en un servidor alojado dentro de la empresa.
\end{itemize}

\subsection{Suposiciones y dependencias}
\begin{itemize}
\item El sistema operativo a usar será Windows 10 para la mayoría de las computadoras, excepto por una que tiene MacOs. También se está suponiendo que hay una conexión a internet estable para todos los teléfonos y computadoras.
\item Las computadoras son todas fabricadas después de 2007, con características bastante variantes.
\end{itemize}

\subsection{Requisitos futuros}
\begin{itemize}
\item La capacidad de generar tickets
\item Acceso a todas las funciones atreves de una versión para dispositivos móviles.
\end{itemize}


%%% Local Variables:
%%% mode: latex
%%% TeX-master: "../srs"
%%% End:
