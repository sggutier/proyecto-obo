En el presente documento se explicarán y analizarán los requisitos del proyecto “Sistema de Inventario de Estudio Fotográfico”, desarrollado para la empresa “Yuriria Foto”. Se adopta la guía de requerimientos de software de la IEEE.

\subsection{Propósito}

Este documento tiene como propósito dar a conocer el funcionamiento general del proyecto SIEF (Sistema de Inventario de Estudio Fotográfico) que está dirigido al equipo desarrollador, a la empresa “Yuriria Foto” y al usuario final.

\subsection{Ámbito del sistema}

\begin{itemize}
\item Nombre del sistema: SIEF, “Sistema de Inventario de Estudio Fotográfico”.
\item El sistema ayudara con el control de los inventarios dentro del estudio fotográfico, manejo del inventario de materiales internos, productos en venta, además de la capacidad de llevar un control de la localización de pedidos, la calendarización de eventos y un control de los empleados con sus horarios.
\item Los principales beneficiados del SIEF será el estudio fotográfico Yuriria Foto y sus clientes, el objetivo es la optimización del tiempo de los procesos administrativos, y la correcta administración de los inventarios, empleados y servicios.
\end{itemize}

\subsection{Definiciones, Acrónimos y Abreviaturas}

\begin{itemize}
\item \textbf{SIEF} Sistema de Inventario de Estudio Fotográfico.
\item \textbf{IEEE} Institute of Electrical and Electronics Engineers.
\item \textbf{NUBE} Almacenamiento en servidores conectados de forma remota.
\item \textbf{BD} Base de datos.
\end{itemize}

% ¿Lo de Referencias si va ahí?
% \subsection{Referencias}

\subsection{Visión general del documento}

El documento está dividido en 4 secciones: 

\begin{itemize}
\item La sección 1 se enfoca en la explicación, objetivos, metas y descripción del documento. 
\item La sección 2 está orientada, como su nombre lo indica, a la descripción general del sistema, donde la información está orientada al cliente/usuario potencial. 
\item La sección 3 trata sobre los requisitos específicos. Se emplean términos técnicos orientados principalmente a los desarrolladores y programadores.
\item La sección 4 son los apéndices, contiene ligas directas al Wiki, foro y podcast de la entrevista, además de una imagen ilustrativa de los componentes del sistema en general.
\end{itemize}


%%% Local Variables:
%%% mode: latex
%%% TeX-master: "../srs"
%%% End:
