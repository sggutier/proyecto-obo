En esta sección se tienen con más detalle los requerimientos específicos del sistema a desarrollar.

\subsection{Interfaces externas}
\begin{itemize}
\item Contar con la librería .NET de Windows en su versión 2018.
\item Contar con Windows 10 con sus últimas actualizaciones.
\end{itemize}

\subsection{Funciones}
\begin{itemize}
\item Sólo permitir el acceso por medio de autentificación del empleado.
\item Permitir agregar, editar, consultar y eliminar registros de los inventarios.
\item Asignar eventos a determinadas fechas para su posterior consulta.
\item Consultar del estado de los empleados, así como su disposición.
\item La asignación de empleados a trabajos o eventos.
\item Solicitar confirmación del usuario cuando se desee editar o eliminar algún registro.
\item Contar con la capacidad de realizar ventas.
\end{itemize}

\subsection{Requisitos de rendimiento}
\begin{itemize}
\item Se podrá acceder al sistema desde cualquier computadora conectada a la red local
  con un navegador actualizado.
\item Soportar 10 usuarios conectados en paralelo.
\item Soportar mínimo 5 modificaciones de la base de datos por segundo.
\item Almacenamiento de más de mil registros por mes.
\end{itemize}

\subsection{Restricciones de diseño}
\begin{itemize}
\item Debido a la solicitud de la empresa Yuriria Foto el diseño debe de ser mayormente compuesto por el color azul.
\item Todas las funciones del sistema deben ser mostradas únicamente en una ventana.
\item Los botones deben ser fácilmente visibles y debe haber una separación entre ellos que minimice el riesgo de oprimir el equivocado por estar muy cerca de otro.
\end{itemize}

\subsection{Atributos del sistema}
\begin{itemize}
\item Para el acceso al sistema se deberá acceder por medio de una verificación que cuente con un usuario y una contraseña.
\item Sólo el usuario administrador tiene permiso a eliminar o editar registros.
\item Sólo el administrador puede crear registros de eventos y asignar a algún empleado a él.
\end{itemize}

\subsubsection{Fiabilidad}
\begin{itemize}
\item El sistema debe tener una interfaz de uso intuitiva y sencilla.
\item La interfaz de usuario debe ajustarse a las características de resolución de los equipos, dentro de la cual estará incorporado el sistema de gestión de procesos y el inventario.
\end{itemize}

\subsubsection{Disponibilidad}

La disponibilidad del sistema debe ser continua con un nivel de servicio para los usuarios de 7 días por 24 horas, contar con una contingencia, generación de alarmas. 


\subsubsection{Portabilidad}

El sistema será implantado con el uso de PHP y HTML5. 


\subsubsection{Mantenibilidad}
\begin{itemize}
\item Nosotros nos encargaremos de mantener el código en un repositorio privado de git para
  posteriormente hacer actualizaciones.
\item La interfaz debe estar complementada con un buen sistema de ayuda (la administración puede recaer en personal con poca experiencia en el uso de aplicaciones informáticas). 
\item En la primera semana de entrega, se dará una capacitación de uso del sistema.  Posteriormente no se dará más instrucciones acerca de su uso.
 \item En caso de haber errores, le daremos nuestro correo electrónico al encargado para que éste reporte errores.  Se permitirán hasta 2 falsas alarmas por semana; en caso de que haya 2 falsas alarmas, ya no se tomarán reportes errores hasta la siguiente semana.
\item Durante el primer año se harán corrección de errores; posteriormente, se le entregará el código fuente al cliente y no nos haremos responsables de ahí en adelante.  En caso de necesitar actualizaciones que amenacen a la estabilidad del sistema (errores críticos), estas se harán durante las siguientes 48 horas de haberse reportado.
\end{itemize}


% \subsection{Otros requisitos}

%%% Local Variables:
%%% mode: latex
%%% TeX-master: "../srs"
%%% End:
